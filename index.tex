% Options for packages loaded elsewhere
\PassOptionsToPackage{unicode}{hyperref}
\PassOptionsToPackage{hyphens}{url}
\PassOptionsToPackage{dvipsnames,svgnames,x11names}{xcolor}
%
\documentclass[
  letterpaper,
  DIV=11,
  numbers=noendperiod]{scrreprt}

\usepackage{amsmath,amssymb}
\usepackage{iftex}
\ifPDFTeX
  \usepackage[T1]{fontenc}
  \usepackage[utf8]{inputenc}
  \usepackage{textcomp} % provide euro and other symbols
\else % if luatex or xetex
  \usepackage{unicode-math}
  \defaultfontfeatures{Scale=MatchLowercase}
  \defaultfontfeatures[\rmfamily]{Ligatures=TeX,Scale=1}
\fi
\usepackage{lmodern}
\ifPDFTeX\else  
    % xetex/luatex font selection
\fi
% Use upquote if available, for straight quotes in verbatim environments
\IfFileExists{upquote.sty}{\usepackage{upquote}}{}
\IfFileExists{microtype.sty}{% use microtype if available
  \usepackage[]{microtype}
  \UseMicrotypeSet[protrusion]{basicmath} % disable protrusion for tt fonts
}{}
\makeatletter
\@ifundefined{KOMAClassName}{% if non-KOMA class
  \IfFileExists{parskip.sty}{%
    \usepackage{parskip}
  }{% else
    \setlength{\parindent}{0pt}
    \setlength{\parskip}{6pt plus 2pt minus 1pt}}
}{% if KOMA class
  \KOMAoptions{parskip=half}}
\makeatother
\usepackage{xcolor}
\setlength{\emergencystretch}{3em} % prevent overfull lines
\setcounter{secnumdepth}{5}
% Make \paragraph and \subparagraph free-standing
\ifx\paragraph\undefined\else
  \let\oldparagraph\paragraph
  \renewcommand{\paragraph}[1]{\oldparagraph{#1}\mbox{}}
\fi
\ifx\subparagraph\undefined\else
  \let\oldsubparagraph\subparagraph
  \renewcommand{\subparagraph}[1]{\oldsubparagraph{#1}\mbox{}}
\fi


\providecommand{\tightlist}{%
  \setlength{\itemsep}{0pt}\setlength{\parskip}{0pt}}\usepackage{longtable,booktabs,array}
\usepackage{calc} % for calculating minipage widths
% Correct order of tables after \paragraph or \subparagraph
\usepackage{etoolbox}
\makeatletter
\patchcmd\longtable{\par}{\if@noskipsec\mbox{}\fi\par}{}{}
\makeatother
% Allow footnotes in longtable head/foot
\IfFileExists{footnotehyper.sty}{\usepackage{footnotehyper}}{\usepackage{footnote}}
\makesavenoteenv{longtable}
\usepackage{graphicx}
\makeatletter
\def\maxwidth{\ifdim\Gin@nat@width>\linewidth\linewidth\else\Gin@nat@width\fi}
\def\maxheight{\ifdim\Gin@nat@height>\textheight\textheight\else\Gin@nat@height\fi}
\makeatother
% Scale images if necessary, so that they will not overflow the page
% margins by default, and it is still possible to overwrite the defaults
% using explicit options in \includegraphics[width, height, ...]{}
\setkeys{Gin}{width=\maxwidth,height=\maxheight,keepaspectratio}
% Set default figure placement to htbp
\makeatletter
\def\fps@figure{htbp}
\makeatother
\newlength{\cslhangindent}
\setlength{\cslhangindent}{1.5em}
\newlength{\csllabelwidth}
\setlength{\csllabelwidth}{3em}
\newlength{\cslentryspacingunit} % times entry-spacing
\setlength{\cslentryspacingunit}{\parskip}
\newenvironment{CSLReferences}[2] % #1 hanging-ident, #2 entry spacing
 {% don't indent paragraphs
  \setlength{\parindent}{0pt}
  % turn on hanging indent if param 1 is 1
  \ifodd #1
  \let\oldpar\par
  \def\par{\hangindent=\cslhangindent\oldpar}
  \fi
  % set entry spacing
  \setlength{\parskip}{#2\cslentryspacingunit}
 }%
 {}
\usepackage{calc}
\newcommand{\CSLBlock}[1]{#1\hfill\break}
\newcommand{\CSLLeftMargin}[1]{\parbox[t]{\csllabelwidth}{#1}}
\newcommand{\CSLRightInline}[1]{\parbox[t]{\linewidth - \csllabelwidth}{#1}\break}
\newcommand{\CSLIndent}[1]{\hspace{\cslhangindent}#1}

\usepackage{amsmath}
\KOMAoption{captions}{tableheading}
\makeatletter
\makeatother
\makeatletter
\@ifpackageloaded{bookmark}{}{\usepackage{bookmark}}
\makeatother
\makeatletter
\@ifpackageloaded{caption}{}{\usepackage{caption}}
\AtBeginDocument{%
\ifdefined\contentsname
  \renewcommand*\contentsname{Table of contents}
\else
  \newcommand\contentsname{Table of contents}
\fi
\ifdefined\listfigurename
  \renewcommand*\listfigurename{List of Figures}
\else
  \newcommand\listfigurename{List of Figures}
\fi
\ifdefined\listtablename
  \renewcommand*\listtablename{List of Tables}
\else
  \newcommand\listtablename{List of Tables}
\fi
\ifdefined\figurename
  \renewcommand*\figurename{Figure}
\else
  \newcommand\figurename{Figure}
\fi
\ifdefined\tablename
  \renewcommand*\tablename{Table}
\else
  \newcommand\tablename{Table}
\fi
}
\@ifpackageloaded{float}{}{\usepackage{float}}
\floatstyle{ruled}
\@ifundefined{c@chapter}{\newfloat{codelisting}{h}{lop}}{\newfloat{codelisting}{h}{lop}[chapter]}
\floatname{codelisting}{Listing}
\newcommand*\listoflistings{\listof{codelisting}{List of Listings}}
\makeatother
\makeatletter
\@ifpackageloaded{caption}{}{\usepackage{caption}}
\@ifpackageloaded{subcaption}{}{\usepackage{subcaption}}
\makeatother
\makeatletter
\@ifpackageloaded{tcolorbox}{}{\usepackage[skins,breakable]{tcolorbox}}
\makeatother
\makeatletter
\@ifundefined{shadecolor}{\definecolor{shadecolor}{rgb}{.97, .97, .97}}
\makeatother
\makeatletter
\makeatother
\makeatletter
\makeatother
\ifLuaTeX
  \usepackage{selnolig}  % disable illegal ligatures
\fi
\IfFileExists{bookmark.sty}{\usepackage{bookmark}}{\usepackage{hyperref}}
\IfFileExists{xurl.sty}{\usepackage{xurl}}{} % add URL line breaks if available
\urlstyle{same} % disable monospaced font for URLs
\hypersetup{
  pdftitle={Financial Anomalies},
  pdfauthor={Krishna Neupane},
  colorlinks=true,
  linkcolor={blue},
  filecolor={Maroon},
  citecolor={Blue},
  urlcolor={Blue},
  pdfcreator={LaTeX via pandoc}}

\title{Financial Anomalies}
\author{Krishna Neupane}
\date{2024-08-24}

\begin{document}
\maketitle
\ifdefined\Shaded\renewenvironment{Shaded}{\begin{tcolorbox}[boxrule=0pt, sharp corners, interior hidden, frame hidden, borderline west={3pt}{0pt}{shadecolor}, enhanced, breakable]}{\end{tcolorbox}}\fi

\renewcommand*\contentsname{Table of contents}
{
\hypersetup{linkcolor=}
\setcounter{tocdepth}{2}
\tableofcontents
}
\bookmarksetup{startatroot}

\hypertarget{preface}{%
\chapter*{Preface}\label{preface}}
\addcontentsline{toc}{chapter}{Preface}

\markboth{Preface}{Preface}

The article is desiged to study financial anomalies

\bookmarksetup{startatroot}

\hypertarget{introduction}{%
\chapter{Introduction}\label{introduction}}

Fama and MacBeth (1973) show two-parameter regression model estimates
average risk-return relationships based on efficient market porfolio
(\(m\)), that is, the market prices fully reflect the available
information. The asset are constructed based on Equation
\ref{Equation:BlackScholes1972} for an asset (\(i\)) proposed by @Black
(1972).

\begin{align}
    &x_{im}  \equiv \frac{\text{total market value of all units of assets} \ i}{\text{total market value of all assets}} \\
    &\text{where asset} (i) \text{in the portfolio} (m) \nonumber 
\end{align}
\label{Equation:BlackScholes1972}

Excepted return of a security (\(i\)) is \(E(\tilde{R_0})\), the
expected return on a security that is riskless in the portfolio \(m\),
plus a risk premium that is \(\beta_i\) times the difference between
expected return of the portfolio (\(E(\tilde{R_m})\) ) and riskless
portfolio (\(E(\tilde{R_0})\)). is calculated by Equation
\ref{Equation:ExpectedReturnFama1953}, \(\beta_i\) is the risk of the
asset \(i\) of the portfolio \(m\), measured relative to
\(\sigma^{2}(\tilde{R}_m)\)

\begin{align}
    &E(\tilde{R_i}) = \left[E(\tilde{R_m}) -S_m \sum\tilde{R_m}  \right] + S_m \sigma (\tilde{R_m}) \beta_i, \nonumber \\
    &\text{where}, \nonumber \\ 
    &\beta_i \equiv \frac{cov(\tilde{R_i}, \tilde{R_m})}{\sigma^2(\tilde{R_m})} = \frac{\sigma_{j=1}^{N} x_{jm}\sigma_{ij}}{\sigma^2 (\tilde{R_m})}=\frac{cov(\tilde{R_i},\tilde{R_m})/\sigma(\tilde{R_m})}{\sigma (\tilde{R_m})} \nonumber \\
    &S_m=\frac{E(\tilde{R_m})-E(\tilde{R_0})}{\sigma (\tilde{R_m})}  \nonumber\\
    \text{hence} \nonumber \\
    &E(\tilde{R_i}) = E(\tilde{R_0}) + \left[ E(\tilde{R_m}) - E(\tilde{R_0}) \right] \beta_i
\end{align}
\label{Equation:ExpectedReturnFama1953}

For each period of \(t\), the cross sectional regression is given by

\begin{align}
    & R_{pt} = \tilde{\gamma}_{0t} + \tilde{\gamma}_{1t} \tilde{\beta}_{p,t-1}+\tilde{\gamma}_{2t} \tilde{\beta}_{p,t-1}^{2} + \tilde{\gamma}_{3t} \bar{s}_{p, t-1} \tilde{\epsilon}_{i}+\tilde{\eta}_{pt}, \\
    &p=1,2,...t \nonumber
\end{align}
\label{Equation:ExpectedReturnFama1953EquationTen}

Equation \ref{Equation:ExpectedReturnFama1953EquationTen} the indepenent
variable \(\tilde{\beta}_{p,t-1}\) is the average of the
\(\tilde{\beta}_i\) for securities in portfolio \(p\),
\(\tilde{\beta}_{p,t-1}^2\) is the average of the squared values of
these \(\tilde{\beta}_i\), \(\bar{s}_{p, t-1} \tilde{\epsilon}_{i}\) is
the average of \(s\tilde{\epsilon}_i\) for portfolio \(p_i\)

Gupta and Ofer (1975) examines investors growth expectations reflected
in the stock prices. A change in the expectation is reflected in the
price movement. The study defines the earnings price ratio is a function
of risk characteristics of the security and the expected growth in the
earnings in Equation \ref{Equation:OferEquation1975EquationOne}. The
risk component are measured by: - the beta coefficient - the firm asset
size (natural logarithm of total asset) - dividend payout ratio -
leverage ratio of liabilties and preferred stocks to the common stock
outstanding - earnings variablity (standard deviation of earnings to
price ratio calcuated over period of seven years)

\begin{align}
    &EP = f(RS, EG) \\
    &\text{where:} \nonumber \\
    & EP = \text{earnings price ratio- }, \nonumber \\
    & RS = \text{risk characteristics of the security} \nonumber \\
    & EG =\text{the expected growth rate in the earnings} \nonumber \\
\end{align}
\label{Equation:OferEquation1975EquationOne}

\begin{align}
    & \Delta P_{i}^{t} = \frac{P_{it}-P_{it-1}}{P_{it-1}} \times 100 \\
    & \text{where:} \nonumber \\ 
    & \Delta P_{i}^{t} \text{is the percent change of the security } i \text{during the period } t-1 \text{to} t \nonumber \\
    & \text{The average yearly percentage of the prices change for portfolio } j (\Delta P_j) \text{ is given by} \nonumber \\
    & \Delta P_{j} =\frac{\sum_{t=1}^{14} \frac{\sum_{r_t=10j-9}^{10j} \Delta P_{rt}^{t}}{14} }{14}, r_{t}=1,\cdots,190 \nonumber\\
    & \text{where:} \nonumber \\
    & r_{t} = \text{the relative ranking of a security at time } t \text{ according to its prediction error at that year}. \\
    & \Delta P_{t}^{rt} = \text{percentage price change during the period } t-1 \text{ to } t \text{ for a security that has the rank of } r_t \text{ at time } t
\end{align}
\label{Equation:OferEquation1975}

Basu (1977): to determine empirically whether the investment performance
of the common stocks is related to the \(P/E\) ratios. \(P/E\) is the
ratio of market value of the common stock (market price times the number
of shares outstanding) as of December 31 to reporting annual earnings
(before extraordinary items) available for common stockholders.
According to the paper, due the exaggerated investor expectations,
\(P/E\) ratio may be the indicator of future investment peroformance.
The paper shows that \(P/E\) is not fully reflected in security prices
in as rapid a manner as postulated by the semi-strong form of the
efficient market hypothesis. Instead, it seems that disequilibrium
persisted in the capital market during the period studied. The results
suggests that market efficiency does not exist due to lags and frictions
in the price adjustmnet process.

Litzenberger and Ramaswamy (1979):

\begin{align}
    & \tilde{R} -r_{ft} = \gamma_{0} +  \gamma_{1} \beta_{it} +  \gamma_{2} (d_{it}-r_{ft}) + \tilde{\epsilon}_{it}, i=1,2,\cdots,N, t=1,2,\cdots,T \\
    &\text{where :} \nonumber \\
    & \tilde{R}_{it} \text{Return of the security } i \text{ in period } t, \beta_{it} \text{ and } d_{it} \text{ are the systemic risk and the dividend yield of security  } i \text{ in period } t. \nonumber \\
    & \tilde{\epsilon}_{it}, a disturbance term is \tilde{R} - E(\tilde{R}_{it}), the deviation of the realized return from the expected value.\nonumber \\
    & \text{coeffiecients } \gamma_0, \gamma_1, \gamma_2
\end{align}
\label{Equation:Litzenberger1979Eqn23}

Banz (1981): studies ``size effect'' by using market indices - Market
Index, CRSP value weighted, CRSP equal weighted indexes, US Bond Index,
Corporate Bond Index to estimate the relationship between the return and
market value based on Equation \ref{Equation:banz1981Equation3}. The
study suggests the CAPM is misspecified.

\begin{align}
    &R_{it} = \gamma_{0t}+\gamma_{1t}\beta_{1t}+\gamma_{2t} \left[ (\phi_{1t}-\phi_{mt}) \right] + \epsilon_{it}, i=1,\cdots,N \\
    &\text{where:} \nonumber \\
    & R_i = \text{return on security } i \nonumber \\
    & \gamma_0= \text{return on a zero-beta portfolio}\nonumber \\
    & \gamma_1= \text{return on market risk premium}\nonumber \\
    & \phi_i= \text{market value security } i\nonumber \\
    & \phi_m= \text{average market value } \nonumber \\
    & \gamma_2 \text{constant measuring the contribution of} \phi_i \text{to the expected return of a security} \nonumber \\
\end{align}
\label{Equation:banz1981Equation3}

Figlewski (1981) estimates ``The test methodology we will use is a
familiar one. We construct ten port? folios to which stocks are assigned
according to their average short interest during the previous six
months, and calculate the portfolio a's or excess returns.''

Basu (1983): The empirical findings reported in this paper indicate
that, at least during the 1963-80 time period, the returns on the common
stock of NYSE firms appear to have been related to earnings' yield and
firm size.

De Bondt and Thaler (1985): ``This study of market efficiency
investigates whether such behavior affects stock prices. The empirical
evidence, based on CRSP monthly return data, is consistent with the
overreaction hypothesis. Substantial weak form market inefficiencies are
discovered. The results also shed new light on the January returns
earned by prior''winners'' and ``losers.'' Portfolios of losers
experience exceptionally large January returns as late as five years
after portfolio formation. To repeat, our goal is to test whether the
overreaction hypothesis is predictive. Specifically, two hypotheses are
suggested: (1) Extreme movements in stock prices will be followed by
subsequent price movements in the opposite direction. (2) The more
extreme the initial price movement, the greater will be the subsequent
adjustment. Both hypotheses imply a violation of weak-form market
efficiency.''

Bhandari (1988): ``The expected returns on common stocks are positively
related to the debt/equity ratio (DER), controlling for the beta and
firm size, both including and excluding January. This relationship is
not sensitive to variations in the market proxy, estimation technique,
etc. The evidence suggests that the''premium'' associated with the
debt/equity ratio is not likely to be just some kind of ``risk
premium''''

\begin{align}
    & E(\tilde{r}_i) = E(\tilde{\gamma}_0) + E(\tilde{\gamma}_1){LTEQ}_i+E(\tilde{\gamma}_2){BETA}_i+E(\tilde{\gamma}_3)+{DER}_i, i=1,\cdots, N \\
    &\text{where: } \nonumber \\
    &\text{Natural Logarithm of Total Common Equity (LETQ) where total common equity} \nonumber \\ &\text{ is the number of shares outstanding times (month-end) price per share }\nonumber \\
    &\text{Debt to Equity Ratio (DER)}\nonumber = \frac{\text{book value of total assets} -\text{book value of common equity}}{\text{market value of common equity}}\\
    &\text{BETA, the risk measure}\nonumber
\end{align}
\label{Equation:bhandari1988debtEquation1}

Amihud and Mendelson (1986): ``llliquidity can be measured by the cost
of immediate execution. An investor willing to transact faces a
tradeoff: He may either wait to transact at a favorable price or insist
on immediate execution at the current bid or ask price. The quoted ask
(offer) price includes a premium for immediate buying, and the bid price
similarly reflects a concession required for immediate sale. Thus. a
natural measure of illiquidity is the spread between the bid and ask
prices, which is the sum of the buying premium and the selling
concession.' Indeed. the relative spread on stocks has been found to be
negatively correlated with liquidity characteristics such as the trading
volume, the number of shareholders. the number of market makers trading
the stock and the stock price continuity.''

Amihud and Mendelson (1989):

The percentage bid-ask spread (= dollar spread divided by the stock
price)

\begin{align}
    &R_{pm} = \gamma_{0}+\gamma_{1}\beta_{pn}+\gamma_{2}\sigma_{pn}+\gamma_{3}SZ_{pn}+\gamma_{4}S_{pn}+\sum_{n=1}^{19}d_n{DY}_n+\epsilon_{pn}
    &\text{where: } \nonumber \\
    & p \text{ is the portolio} \nonumber \\
    & \beta_{pn} \text{ portfolio beta} \nonumber \\
    & \sigma_{pn} \text{ residual standard deviation} \nonumber \\
    & S_{pn} \text{ the portfolio average spread} \nonumber \\
    & {SZ}_{pn} \text{ average market value of the portfolio} \nonumber \\
\end{align}
\label{Equation:amihud1989effectsEquation2}

Ou and Penman (1989): ``Implements multivariate LOGIT analysis in
financial statements. Fundamental analysis extracts value measures from
financial statements and compares them to prices to identify mispriced
stocks. The evidence here suggests that financial statements capture
fundamentals that are not reflected in prices.''

Jegadeesh (1990): ``This paper documents strong evidence of predictable
behavior of security returns. The results here show that the monthly
returns on individual stocks exhibit. significantly negative first-order
serial correlation and significantly positive higher-order serial
correlation. The pattern of serial correlation exhibits seasonality,
with the pattern in January significantly different from that in the
other months.''

Jegadeesh and Titman (1993):

Loughran and Ritter (1995) : ``Investing in firms issuing stock is
hazardous to your wealth. Firms issuing stock during 1970 to 1990,
whether an IPO or an SEO, have been poor long-run investments for
investors. s. The average annual return during the five years after
issuing is only 5 percent for firms conducting IPOs, and only 7 percent
for firms conducting SEOs. Investing an equal amount at the same time in
a nonissuing firm with approximately the same market capitalization, and
holding it for an identical period, would have produced an average
compound return of 12 percent per year for IPOs and 15 percent for SEOs.
The magnitude of the underperformance is large: it implies that 44
percent more money would need to be invested in the issuers than in the
nonissuers to be left with the same wealth five years later.''

Michaely, Thaler, and Womack (1995): ``Market reaction to the dividend
initiations.''

La Porta (1996): '' Contrarian strategies that use analysts'
expectations to form portfolios yield high returns. Specifically, when
stocks are sorted by the expected growth rate in earnings, low E\{g\}
stocks beat high E\{g\} stocks by twenty percentage points.. Finally,
event study evidence suggests that the market was overly pessimistic
about the earnings of the low E\{g\} portfolio and excessively
optimistic about the earnings of the high E\{g\} portfolio''

Lev and Sougiannis (1996): ``To address these concerns, we estimate the
R\&D capital of a large sample of public companies and find these
estimates to be statistically reliable and economically meaningful. We
then adjust the reported earnings and book values of sample firms for
the R\&D capitalization and find that such adjustments are
value-relevant to investors.''

Sloan (1996): ``This paper investigates whether stock prices reflect
information about future earnings contained in the accrual and cash flow
components of current earnings. The persistence of earnings performance
is shown to depend on the relative magnitudes of the cash and accrual
components of earnings. However, stock prices act as if investors fail
to identify correctly the different properties of these two components
of earnings''

Womack (1996): '' An analysis of new buy and sell recommendations of
stocks by security analysts at major U.S. brokerage firms shows
significant, systematic discrepancies between prerecommendation prices
and eventual values. The initial return at the time of the
recommendations is large, even though few recommendations coincide with
new public news or provide previously unavailable facts. However, these
initial price reactions are incomplete. For buy recommendations, the
mean postevent drift is modest (+2.4\%) and short-lived, but for sell
recommendations, the drift is larger (-9.1\%) and extends for six
months. Analysts appear to have market timing and stock picking
abilities''

Brennan, Chordia, and Subrahmanyam (1998): ``We examine the relation
between stock returns, measures of risk, and several non-risk security
characteristics, including the book-to-market ratio, firm size, the
stock price, the dividend yield, and lagged returns. Our primary
objective is to determine whether non-risk characteristics have marginal
explanatory power relative to the arbitrage pricing theory benchmark,
with factors determined using, in turn, the Connor and Korajczyk (CK;
1988) and the Fama and French (FF; 1993b) approaches.
Fama---MacBeth-type regressions using risk adjusted returns provide
evidence of return momentum, size, and book-to-market effects, together
with a significant and negative relation between returns and trading
volume, even after accounting for the CK factors. When the analysis is
repeated using the FF factors, we find that the size and book-to-market
effects are attenuated, while the momentum and trading volume effects
persist. In addition, Nasdaq stocks show significant underperformance
after adjusting for risk using either method''

Brennan, Chordia, and Subrahmanyam (1998): Regardless of the method used
to risk-adjust returns, we find a strong negative relation between
average returns and trading volume, which is consistent with a liquidity
premium in asset prices. In addition, the size and book-to-market ratio
effects are strong in the CK method of risk-adjustment, while the FF
factors attenuate both the magnitude and significance of these effects.

Dichev (1998): '' Several studies suggest that a firm distress risk
factor could be behind the size and the book-to-market effects. A
natural proxy for firm distress is bankruptcy risk. If bankruptcy risk
is systematic, one would expect a positive association between
bankruptcy risk and subsequent realized returns. However, resullts
demonstrate that bankruptcy risk is not rewarded by higher returns.
Thus, a distress factor is unlikely to account for the size and
book-to-market effects. Surprisingly, firms with high bankruptcy risk
earn lower than average returns since 1980. A risk-based explanation
cannot fully explain the anomalous evidence.''

\begin{itemize}
\tightlist
\item
  MV is log of fiscal-year-end price times number of shares outstanding
\item
  BIM is common equity divided by fiscal-year-end price times number of
  shares outstandin
\item
  Z risk (bankruptcy risk, Altman (1968)) comprised of

  \begin{itemize}
  \tightlist
  \item
    Working capital/Total assets
  \item
    Retained Earnings/Total assets
  \item
    Earnings before interest and taxes/Total assets
  \item
    Market value equity/Book value of total debt
  \item
    Sales/Total assets
  \end{itemize}
\item
  O risk

  \begin{itemize}
  \item
    SIZE = log(total assets/GNP price-level index).
  \item
    \begin{itemize}
    \tightlist
    \item
      TLTA = Total liabilities divided by total assets.
    \end{itemize}
  \item
    WCTA Working capital divided by total assets.
  \item
    CLCA Current liabilities divided by current assets.
  \item
    OENEG = One if total liabilities exceeds total assets, zero
    otherwise.
  \item
    NITA Net income divided by total assets.
  \item
    FUTL = Funds provided by operations divided by total liabilities.
  \item
    INTWO = One if net income was negative for the last two years, zero
    otherwise.
  \item
    CHIN = \(({NI}_t-{NI}_{t-1})/(|{NI}_t| + |{NI}_{t-1}|)\) where
    \({NI}_t\) is net income for the most recent period.
  \end{itemize}
\end{itemize}

Datar, Naik, and Radcliffe (1998): ``In this paper, we provide an
alternative test of A\&M's model using the turnover rate as a proxy for
liquidity and found strong support for A\&M's model. In particular, we
find that the stock returns are strongly negatively related to their
turnover rates confirming the notion that illiquid stocks provide higher
average returns.'' - Monthly Returns - turnover rate of every stock =
monthly trading volume (the average number of shares traded during the
previous three months, i.e., during months t- 3, t-2 and t-1) and divide
it by the number of shares outstanding of that firm - turnover rate of
every stock - book-to-market ratio, - firm size - firm beta

Moskowitz and Grinblatt (1999): '' This paper documents a strong and
prevalent momentum effect in ind ponents of stock returns which accounts
for much of the individual sto tum anomaly. Specifically, momentum
investment strategies, which buy p stocks and sell past losing stocks,
are significantly less profitable once for industry momentum. By
contrast, industry momentum investmen which buy stocks from past winning
industries and sell stocks from p industries, appear highly profitable,
even after controlling for size, book equity, individual stock momentum,
the cross-sectional dispersion in m and potential microstructur''

Lee and Swaminathan (2000): ``This study shows that past trading volume
provides an important link between''momentum'' and ``value'' strategies.
Specifically, we find that firms with high (low) past turnover ratios
exhibit many glamour (value) characteristics, earn lower (higher) future
returns, and have consistently more negative (positive) earnings
surprises over the next eight quarters. Past trading volume also
predicts both the magnitude and persistence of price momentum.
Specifically, price momentum effects reverse over the next five years,
and high (low) volume winners (losers) experience faster reversals.
Collectively, our findings show that past volume helps to reconcile
intermediate-horizon ``underreaction'' and long-horizon ``overreaction''
effects''

Asness, Porter, and Stevens (2000): ``Better proxies for the information
about future returns contained in firm characteristics such as size,
book-to-market equity, cash flow-to-price, percent change in employees,
and various past return measures are obtained by breaking these
explanatory variables into two industry-related components.''

Piotroski (2000):

Chordia, Subrahmanyam, and Anshuman (2001): ``A body of literature
starting with Amihud and Mendelson (1986) has found that investors
demand a premium for less liquid stocks, so that expected returns.
should be negatively related to the level of liquidity. In this paper,
we document negative and signi''cant cross-sectional relationship
between average stock returns and the level as well as the second moment
of measures of trading activity such as dollar volume and share
turnover. Given the evidence that the level of liquidity affects asset
returns, a reasonable hypothesis is that the second moment of liquidity
should be positively related to asset returns, provided agents care
about the risk associated with fluctuations in liquidity. Motivated by
this observation, we analyze the relation between expected equity
returns and the level as well as the volatility of trading activity, a
proxy for liquidity. We document a result contrary to our initial
hypothesis, namely, a negative and surprisingly strong cross-sectional
relationship between stock returns and the variability of dollar trading
volume and share turnover, after controlling for size, book-to-market
ratio, momentum, and the level of dollar volume or share turnover. This
e!ect survives a number of robustness checks, and is statistically and
economically signi''cant. Our analysis demonstrates the importance of
trading activity-related variables in the crosssection of expected stock
returns.''

Lamont, Polk, and Saaá-Requejo (2001): ``We test whether the impact of
financial constraints on firm value is observable in stock returns. We
form portfolios of firms based on observable characteristics related to
financial constraints and test for common variation in stock returns.
Financially constrained firms' stock returns move together over time,
suggesting that constrained firms are subject to common shocks.
Constrained firms have low average stock returns in our 1968--1997
sample of growing manufacturing firms. We find no evidence that the
relative performance of constrained firms reflects monetary policy,
credit conditions, or business cycles. We construct various zero-cost
portfolios that are long financially constrained firms and short less
constrained firms and find three results. First, these portfolios
capture common variation in stock returns not captured by other sources
of return comovements. Thus we conclude that there is a financial
constraints factor, an identifiable independent common source of
economic shocks to firm value. The evidence suggests that financial
constraints do affect firm value and that the severity of constraints
varies over time. Second, our investigation of the role of financial
constraints in asset pricing reveals the surprising result that
constrained firms earn lower returns than unconstrained firms, a result
not explainable using existing asset-pricing models. Third, financially
constrained firms do not have returns that are significantly more
cyclical than average. Thus, the source of the common economic shocks to
financially constrained firms remains an open question.The proxies are
constructed based on Kaplan and Zingales (1995)''

Elgers, Lo, and Pfeiffer Jr (2001): ``The paper documents that the
weighthing of analysts annual earnings forecasts implicit in security
prices is lower than the historical relation between the financial
analysts forecasts and realized earnings. Our evidence that analysts the
beginging of the year annual earnings forecasts are associated with
abnormal security returns subsequently accumulated over the earnings
year is consistent with the delayed price reaction to the value-relevant
information in the posistions in the securityies in the bottom (top)
deciles of the corss-sectional distribution of the analysts earnings
forecasts early in the earnings year, generates statistically
siginificant trading profits in the year after porfolio formation for
firms with relatively low analysts coverage.''

P. A. Gompers and Metrick (2001): ``We analyze the investors preferences
for the stock and the implications that these preferences have for
stock-market prices and returns. We find that large institutional
investors- a category including all managers with greater than \$100
million under discretional control --have nearly doubled their share of
the common staock from 1980 to 1996, with most of this increase driven
by the growth in the holdings of the largest one hundred institutions.''

Griffin and Lemmon (2002): ``This paper examines the relationship
between book-to-market equity, distress risk, and stock returns. Among
firms with the highest distress risk as proxied by Ohlson (1980)
0-score, the difference in returns between high and low book-tomarket
securities is more than twice as large as that in other firms. This
large return differential cannot be explained by the three-factor model
or by differences in economic fundamentals. Consistent with mispricing
arguments, firms with high distress risk exhibit the largest return
reversals around earnings announcements, and the book-to-market effect
is largest in small firms with low analyst coverage.''

Diether, Malloy, and Scherbina (2002): ``We provide evidence that stocks
with higher dispersion in analysts' earnings fore-casts earn lower
future returns than otherwise similar stocks. This effect is
mostpronounced in small stocks and stocks that have performed poorly
over the pastyear. Interpreting dispersion in analysts' forecasts as a
proxy for differences inopinion about a stock, we show that this
evidence is consistent with the hypothesisthat prices will reflect the
optimistic view whenever investors with the lowestvaluations do not
trade. By contrast, our evidence is inconsistent with a view
thatdispersion in analysts' forecasts proxies for risk.''

J. Chen, Hong, and Stein (2002): '' In this paper, we bring new evidence
to bear on an asset-pricing hypothesis which has been around for a long
while, but which has thus far not recieved much empirical support. The
idea, which dates back to Miller, has to do with the combined effects of
short-sales constraints and differences of opinion on stock prices. We
develop a model of stock prices in which there are both differences of
opinion among investors as well as short-sales constraints. The key
insights that emerge in that breadth of ownerwhip is a valuation
indicator. When the breadth is low- when investors have long positions
in the stock- this signals that hte short-sales constrarint in binding
tightly, implying that prices are high relative to fundamentls and that
expected reutrns are therefore low.''

P. Gompers, Ishii, and Metrick (2003): ``Shareholder rights vary across
firms. Using the incidence of 24 governance rules, we construct a
``Governance Index'' to proxy for the level of shareholder rights at
about 1500 large firms during the 1990s. An investment strategy that
bought firms in the lowest decile of the index (strongest rights) and
sold firms in the highest decile of the index (weakest rights) would
have earned abnormal returns of 8.5 percent per year during the sample
period. We find that firms with stronger shareholder rights had higher
firm value, higher profits, higher sales growth, lower capital
expenditures, and made fewer corporate acquisitions.''

Doyle, Lundholm, and Soliman (2003): ``We investigate the informational
properties of pro forma earnings. This increasingly popular measure of
earnings excludes certain expenses that the company deems non-recurring,
non-cash, or otherwise unimportant for understanding the future value of
the firm. We find, however, that these expenses are far from
unimportant. Higher levels of exclusions lead to predictably lower
future cash flows. We also find that investors do not fully appreciate
the lower cash flow implications at the time of the earnings
announcement. A trading strategy based on the excluded expenses yields a
large positive abnormal return in the years following the announcement,
and persists after controlling for various risk factors and other
anomalies.''

Doyle, Lundholm, and Soliman (2003):

Watkins (2003): ``I analyze the degree to which return consistency in
the past predicts future returns. It is discovered here that consistency
is a strong predictive measure for future stock returns. In a portfolio
context, positively consistent stocks exhibit positive future
risk-adjusted returns, and negatively consistent stocks exhibit negative
future risk-adjusted returns. The results are economically and
statistically significant over multiple sub-periods. Also, odd return
behavior persists for nearly two years after portfolio formation. Stocks
that have been consistently positive (negative) for longer time horizons
have higher(lower) risk-adjusted returns during the following month than
those which have been consistent for shorter time periods. Finally, it
is determined that high consistency enhances momentum when the two
factors are allowed to interact. Thus, there appears to be strong path
dependence in the momentum effect, and consistency in stock returns
appears to be an important component of return predictability.''

Eberhart, Maxwell, and Siddique (2004): ``We examine a sample of 8,313
cases, between 1951 and 2001, where firms unex edly increase their
research and development (R\&D) expenditures by a signif amount. We find
consistent evidence of a misreaction, as manifested in the si cantly
positive abnormal stock returns that our sample firms' shareholders expe
following these increases. We also find consistent evidence that our
sample firm perience significantly positive long-term abnormal operating
performance follow their R\&D increases. Our findings suggest that R\&D
increases are beneficial in ments, and that the market is slow to
recognize the extent of this benefit (cons with investor
underreaction).'' George and Hwang (2004): ``When coupled with a stock's
current price, a readily available piece of information 52-week high
price-explains a large portion of the profits from momentum investi
Nearness to the 52-week high dominates and improves upon the forecasting
power past returns (both individual and industry returns) for future
returns. Future retu forecast using the 52-week high do not reverse in
the long run. These results indi that short-term momentum and long-term
reversals are largely separate phenome which presents a challenge to
current theory that models these aspects of secur returns as integrated
components of the market's resp''

Jegadeesh et al. (2004)

Titman, Wei, and Xie (2004)

Cremers and Nair (2005)

Acharya and Pedersen (2005):

Measure of Illiquidity (based on Amihud, 2002):

\begin{align}
&ILLIQ_t^t=\frac{1}{Days}_i^t \sum_{d=1}^{Days_t^i} \frac{|R_{td}^i|}{V_{td}^i}
\end{align}
\label{Equation:acharya2005assetEquation11}

Hou and Moskowitz (2005) Nagel (2005) Asquith, Pathak, and Ritter (2005)
Mohanram (2005) Whited and Wu (2006) Ang et al. (2006) Anderson and
Garcia-Feijoo (2006) Daniel and Titman (2006) Fama and French (2006)
Bradshaw, Richardson, and Sloan (2006) Franzoni and Marin (2006)
Narayanamoorthy (2006) Avramov et al. (2007) Kumar et al. (2008) Guo and
Savickas (2008) Campbell, Hilscher, and Szilagyi (2008) Garlappi, Shu,
and Yan (2008) Cooper, Gulen, and Schill (2008) Pontiff and Woodgate
(2008) Cohen and Frazzini (2008) Fabozzi, Ma, and Oliphant (2008)
Soliman (2008) Avramov et al. (2009) Hahn and Lee (2005) Rozeff and
Zaman (1988) Lebedeva, Maug, and Schneider (2012) Fishman and Hagerty
(1995) Cao, Field, and Hanka (2004) Pagano and R"oell (1996) Chemmanur
and Yan (2019) Gow and Taylor (2009) Dechow and Sloan (1997) Tuzel
(2010) Menzly and Ozbas (2010) Xing, Zhang, and Zhao (2010) Bali,
Cakici, and Whitelaw (2011) Yan (2011) Eisfeldt and Papanikolaou (2013)
Callen, Khan, and Lu (2013) Drake, Rees, and Swanson (2011) Hafzalla,
Lundholm, and Matthew Van Winkle (2011) Landsman et al. (2011) Li (2011)
Thomas and Zhang (2011) Johnson and So (2012) Palazzo (2012) Bazdrech,
Belo, and Lin (2009) Cohen and Lou (2012) Hirshleifer, Hsu, and Li
(2013) Prakash and Sinha (2013) Cohen, Diether, and Malloy (2013)
Novy-Marx (2013) Frazzini and Pedersen (2014) Valta (2016) Da and
Warachka (2009) Hou, Karolyi, and Kho (2011) Barber et al. (2001) Harvey
and Siddique (2000) Spiess and Affleck-Graves (1999) Ang, Chen, and Xing
(2006) Chordia and Shivakumar (2006) Kishore et al. (2008) Hawkins,
Chamberlin, and Daniel (1984) Adrian, Etula, and Muir (2014) Frankel and
Lee (1998) Abarbanell and Bushee (1998) Ali, Hwang, and Trombley (2003)
Vanden (2006) Vanden (2004) Chan, Chen, and Hsieh (1985) Arbel, Carvell,
and Strebel (1983) Campbell (1996) Dittmar (2002) Liu (2006) P'astor and
Stambaugh (2003) Hirshleifer and Jiang (2010) Black (1972) Carhart
(1997) Black (1972) Kraus and Litzenberger (1976) Lintner (1965) Sharpe
(1964) Heckerman (1972) Heckerman (1972) Fama and MacBeth (1973) Mossin
(1966) Jagannathan and Wang (1996) Ferson and Harvey (1991) Fogler,
John, and Tipton (1981) Boudoukh et al. (2007) Hou, Xue, and Zhang
(2015) Holthausen and Larcker (1992) Easley, Hvidkjaer, and O'hara
(2010) Chopra, Lakonishok, and Ritter (1992) Kelly, Moskowitz, and
Pruitt (2021) L. Chen, Novy-Marx, and Zhang (2011) Foster, Olsen, and
Shevlin (1984) Spiess and Affleck-Graves (1995) Daniel and Titman (1997)
Fama and French (1993) Fama and French (1992) Lo and Wang (2006) Mayshar
(1981) Brennan and Subrahmanyam (1996) Jiang, Lee, and Zhang (2005) Lev,
Sarath, and Sougiannis (2005)

\bookmarksetup{startatroot}

\hypertarget{summary}{%
\chapter{Summary}\label{summary}}

In summary, this book has no content whatsoever.

\bookmarksetup{startatroot}

\hypertarget{references}{%
\chapter*{References}\label{references}}
\addcontentsline{toc}{chapter}{References}

\markboth{References}{References}

\hypertarget{refs}{}
\begin{CSLReferences}{1}{0}
\leavevmode\vadjust pre{\hypertarget{ref-abarbanell1998abnormal}{}}%
Abarbanell, Jeffery S, and Brian J Bushee. 1998. {``Abnormal Returns to
a Fundamental Analysis Strategy.''} \emph{Accounting Review}, 19--45.

\leavevmode\vadjust pre{\hypertarget{ref-acharya2005asset}{}}%
Acharya, Viral V, and Lasse Heje Pedersen. 2005. {``Asset Pricing with
Liquidity Risk.''} \emph{Journal of Financial Economics} 77 (2):
375--410.

\leavevmode\vadjust pre{\hypertarget{ref-adrian2014financial}{}}%
Adrian, Tobias, Erkko Etula, and Tyler Muir. 2014. {``Financial
Intermediaries and the Cross-Section of Asset Returns.''} \emph{The
Journal of Finance} 69 (6): 2557--96.

\leavevmode\vadjust pre{\hypertarget{ref-ali2003arbitrage}{}}%
Ali, Ashiq, Lee-Seok Hwang, and Mark A Trombley. 2003. {``Arbitrage Risk
and the Book-to-Market Anomaly.''} \emph{Journal of Financial Economics}
69 (2): 355--73.

\leavevmode\vadjust pre{\hypertarget{ref-altman1968financial}{}}%
Altman, Edward I. 1968. {``Financial Ratios, Discriminant Analysis and
the Prediction of Corporate Bankruptcy.''} \emph{The Journal of Finance}
23 (4): 589--609.

\leavevmode\vadjust pre{\hypertarget{ref-amihud1986asset}{}}%
Amihud, Yakov, and Haim Mendelson. 1986. {``Asset Pricing and the
Bid-Ask Spread.''} \emph{Journal of Financial Economics} 17 (2):
223--49.

\leavevmode\vadjust pre{\hypertarget{ref-amihud1989effects}{}}%
---------. 1989. {``The Effects of Beta, Bid-Ask Spread, Residual Risk,
and Size on Stock Returns.''} \emph{The Journal of Finance} 44 (2):
479--86.

\leavevmode\vadjust pre{\hypertarget{ref-anderson2006empirical}{}}%
Anderson, Christopher W, and Luis Garcia-Feijoo. 2006. {``Empirical
Evidence on Capital Investment, Growth Options, and Security Returns.''}
\emph{The Journal of Finance} 61 (1): 171--94.

\leavevmode\vadjust pre{\hypertarget{ref-ang2006downside}{}}%
Ang, Andrew, Joseph Chen, and Yuhang Xing. 2006. {``Downside Risk.''}
\emph{The Review of Financial Studies} 19 (4): 1191--1239.

\leavevmode\vadjust pre{\hypertarget{ref-ang2006cross}{}}%
Ang, Andrew, Robert J Hodrick, Yuhang Xing, and Xiaoyan Zhang. 2006.
{``The Cross-Section of Volatility and Expected Returns.''} \emph{The
Journal of Finance} 61 (1): 259--99.

\leavevmode\vadjust pre{\hypertarget{ref-arbel1983giraffes}{}}%
Arbel, Avner, Steven Carvell, and Paul Strebel. 1983. {``Giraffes,
Institutions and Neglected Firms.''} \emph{Financial Analysts Journal}
39 (3): 57--63.

\leavevmode\vadjust pre{\hypertarget{ref-asness2000predicting}{}}%
Asness, Clifford S, R Burt Porter, and Ross L Stevens. 2000.
{``Predicting Stock Returns Using Industry-Relative Firm
Characteristics.''} \emph{Available at SSRN 213872}.

\leavevmode\vadjust pre{\hypertarget{ref-asquith2005short}{}}%
Asquith, Paul, Parag A Pathak, and Jay R Ritter. 2005. {``Short
Interest, Institutional Ownership, and Stock Returns.''} \emph{Journal
of Financial Economics} 78 (2): 243--76.

\leavevmode\vadjust pre{\hypertarget{ref-avramov2007momentum}{}}%
Avramov, Doron, Tarun Chordia, Gergana Jostova, and Alexander Philipov.
2007. {``Momentum and Credit Rating.''} \emph{The Journal of Finance} 62
(5): 2503--20.

\leavevmode\vadjust pre{\hypertarget{ref-avramov2009dispersion}{}}%
---------. 2009. {``Dispersion in Analysts' Earnings Forecasts and
Credit Rating.''} \emph{Journal of Financial Economics} 91 (1): 83--101.

\leavevmode\vadjust pre{\hypertarget{ref-bali2011maxing}{}}%
Bali, Turan G, Nusret Cakici, and Robert F Whitelaw. 2011. {``Maxing
Out: Stocks as Lotteries and the Cross-Section of Expected Returns.''}
\emph{Journal of Financial Economics} 99 (2): 427--46.

\leavevmode\vadjust pre{\hypertarget{ref-banz1981relationship}{}}%
Banz, Rolf W. 1981. {``The Relationship Between Return and Market Value
of Common Stocks.''} \emph{Journal of Financial Economics} 9 (1): 3--18.

\leavevmode\vadjust pre{\hypertarget{ref-barber2001can}{}}%
Barber, Brad, Reuven Lehavy, Maureen McNichols, and Brett Trueman. 2001.
{``Can Investors Profit from the Prophets? Security Analyst
Recommendations and Stock Returns.''} \emph{The Journal of Finance} 56
(2): 531--63.

\leavevmode\vadjust pre{\hypertarget{ref-basu1977investment}{}}%
Basu, Sanjoy. 1977. {``Investment Performance of Common Stocks in
Relation to Their Price-Earnings Ratios: A Test of the Efficient Market
Hypothesis.''} \emph{The Journal of Finance} 32 (3): 663--82.

\leavevmode\vadjust pre{\hypertarget{ref-basu1983relationship}{}}%
---------. 1983. {``The Relationship Between Earnings' Yield, Market
Value and Return for NYSE Common Stocks: Further Evidence.''}
\emph{Journal of Financial Economics} 12 (1): 129--56.

\leavevmode\vadjust pre{\hypertarget{ref-bazdrech2009labor}{}}%
Bazdrech, Santiago, Frederico Belo, and Xiaoji Lin. 2009. {``Labor
Hiring, Investment and Stock Return Predictability in the Cross
Section.''} \emph{Financial Markets Group, The London School of
Economics and Political Science}.

\leavevmode\vadjust pre{\hypertarget{ref-bhandari1988debt}{}}%
Bhandari, Laxmi Chand. 1988. {``Debt/Equity Ratio and Expected Common
Stock Returns: Empirical Evidence.''} \emph{The Journal of Finance} 43
(2): 507--28.

\leavevmode\vadjust pre{\hypertarget{ref-black1972capital}{}}%
Black, Fischer. 1972. {``Capital Market Equilibrium with Restricted
Borrowing.''} \emph{The Journal of Business} 45 (3): 444--55.

\leavevmode\vadjust pre{\hypertarget{ref-boudoukh2007importance}{}}%
Boudoukh, Jacob, Roni Michaely, Matthew Richardson, and Michael R
Roberts. 2007. {``On the Importance of Measuring Payout Yield:
Implications for Empirical Asset Pricing.''} \emph{The Journal of
Finance} 62 (2): 877--915.

\leavevmode\vadjust pre{\hypertarget{ref-bradshaw2006relation}{}}%
Bradshaw, Mark T, Scott A Richardson, and Richard G Sloan. 2006. {``The
Relation Between Corporate Financing Activities, Analysts' Forecasts and
Stock Returns.''} \emph{Journal of Accounting and Economics} 42 (1-2):
53--85.

\leavevmode\vadjust pre{\hypertarget{ref-brennan1998alternative}{}}%
Brennan, Michael J, Tarun Chordia, and Avanidhar Subrahmanyam. 1998.
{``Alternative Factor Specifications, Security Characteristics, and the
Cross-Section of Expected Stock Returns.''} \emph{Journal of Financial
Economics} 49 (3): 345--73.

\leavevmode\vadjust pre{\hypertarget{ref-brennan1996market}{}}%
Brennan, Michael J, and Avanidhar Subrahmanyam. 1996. {``Market
Microstructure and Asset Pricing: On the Compensation for Illiquidity in
Stock Returns.''} \emph{Journal of Financial Economics} 41 (3): 441--64.

\leavevmode\vadjust pre{\hypertarget{ref-callen2013accounting}{}}%
Callen, Jeffrey L, Mozaffar Khan, and Hai Lu. 2013. {``Accounting
Quality, Stock Price Delay, and Future Stock Returns.''}
\emph{Contemporary Accounting Research} 30 (1): 269--95.

\leavevmode\vadjust pre{\hypertarget{ref-campbell1996understanding}{}}%
Campbell, John Y. 1996. {``Understanding Risk and Return.''}
\emph{Journal of Political Economy} 104 (2): 298--345.

\leavevmode\vadjust pre{\hypertarget{ref-campbell2008search}{}}%
Campbell, John Y, Jens Hilscher, and Jan Szilagyi. 2008. {``In Search of
Distress Risk.''} \emph{The Journal of Finance} 63 (6): 2899--939.

\leavevmode\vadjust pre{\hypertarget{ref-cao2004does}{}}%
Cao, Charles, Laura Casares Field, and Gordon Hanka. 2004. {``Does
Insider Trading Impair Market Liquidity? Evidence from IPO Lockup
Expirations.''} \emph{Journal of Financial and Quantitative Analysis} 39
(1): 25--46.

\leavevmode\vadjust pre{\hypertarget{ref-carhart1997persistence}{}}%
Carhart, Mark M. 1997. {``On Persistence in Mutual Fund Performance.''}
\emph{The Journal of Finance} 52 (1): 57--82.

\leavevmode\vadjust pre{\hypertarget{ref-chan1985exploratory}{}}%
Chan, Kevin C, Nai-fu Chen, and David A Hsieh. 1985. {``An Exploratory
Investigation of the Firm Size Effect.''} \emph{Journal of Financial
Economics} 14 (3): 451--71.

\leavevmode\vadjust pre{\hypertarget{ref-chemmanur2019advertising}{}}%
Chemmanur, Thomas J, and An Yan. 2019. {``Advertising, Attention, and
Stock Returns.''} \emph{Quarterly Journal of Finance} 9 (03): 1950009.

\leavevmode\vadjust pre{\hypertarget{ref-chen2002breadth}{}}%
Chen, Joseph, Harrison Hong, and Jeremy C Stein. 2002. {``Breadth of
Ownership and Stock Returns.''} \emph{Journal of Financial Economics} 66
(2-3): 171--205.

\leavevmode\vadjust pre{\hypertarget{ref-chen2011alternative}{}}%
Chen, Long, Robert Novy-Marx, and Lu Zhang. 2011. {``An Alternative
Three-Factor Model.''} \emph{Available at SSRN 1418117}.

\leavevmode\vadjust pre{\hypertarget{ref-chopra1992measuring}{}}%
Chopra, Navin, Josef Lakonishok, and Jay R Ritter. 1992. {``Measuring
Abnormal Performance: Do Stocks Overreact?''} \emph{Journal of Financial
Economics} 31 (2): 235--68.

\leavevmode\vadjust pre{\hypertarget{ref-chordia2006earnings}{}}%
Chordia, Tarun, and Lakshmanan Shivakumar. 2006. {``Earnings and Price
Momentum.''} \emph{Journal of Financial Economics} 80 (3): 627--56.

\leavevmode\vadjust pre{\hypertarget{ref-chordia2001trading}{}}%
Chordia, Tarun, Avanidhar Subrahmanyam, and V Ravi Anshuman. 2001.
{``Trading Activity and Expected Stock Returns.''} \emph{Journal of
Financial Economics} 59 (1): 3--32.

\leavevmode\vadjust pre{\hypertarget{ref-cohen2013misvaluing}{}}%
Cohen, Lauren, Karl Diether, and Christopher Malloy. 2013. {``Misvaluing
Innovation.''} \emph{The Review of Financial Studies} 26 (3): 635--66.

\leavevmode\vadjust pre{\hypertarget{ref-cohen2008economic}{}}%
Cohen, Lauren, and Andrea Frazzini. 2008. {``Economic Links and
Predictable Returns.''} \emph{The Journal of Finance} 63 (4):
1977--2011.

\leavevmode\vadjust pre{\hypertarget{ref-cohen2012complicated}{}}%
Cohen, Lauren, and Dong Lou. 2012. {``Complicated Firms.''}
\emph{Journal of Financial Economics} 104 (2): 383--400.

\leavevmode\vadjust pre{\hypertarget{ref-cooper2008asset}{}}%
Cooper, Michael J, Huseyin Gulen, and Michael J Schill. 2008. {``Asset
Growth and the Cross-Section of Stock Returns.''} \emph{The Journal of
Finance} 63 (4): 1609--51.

\leavevmode\vadjust pre{\hypertarget{ref-cremers2005governance}{}}%
Cremers, KJ Martijn, and Vinay B Nair. 2005. {``Governance Mechanisms
and Equity Prices.''} \emph{The Journal of Finance} 60 (6): 2859--94.

\leavevmode\vadjust pre{\hypertarget{ref-da2009cashflow}{}}%
Da, Zhi, and Mitchell Craig Warachka. 2009. {``Cashflow Risk, Systematic
Earnings Revisions, and the Cross-Section of Stock Returns.''}
\emph{Journal of Financial Economics} 94 (3): 448--68.

\leavevmode\vadjust pre{\hypertarget{ref-daniel1997evidence}{}}%
Daniel, Kent, and Sheridan Titman. 1997. {``Evidence on the
Characteristics of Cross Sectional Variation in Stock Returns.''}
\emph{The Journal of Finance} 52 (1): 1--33.

\leavevmode\vadjust pre{\hypertarget{ref-daniel2006market}{}}%
---------. 2006. {``Market Reactions to Tangible and Intangible
Information.''} \emph{The Journal of Finance} 61 (4): 1605--43.

\leavevmode\vadjust pre{\hypertarget{ref-datar1998liquidity}{}}%
Datar, Vinay T, Narayan Y Naik, and Robert Radcliffe. 1998. {``Liquidity
and Stock Returns: An Alternative Test.''} \emph{Journal of Financial
Markets} 1 (2): 203--19.

\leavevmode\vadjust pre{\hypertarget{ref-de1985does}{}}%
De Bondt, Werner FM, and Richard Thaler. 1985. {``Does the Stock Market
Overreact?''} \emph{The Journal of Finance} 40 (3): 793--805.

\leavevmode\vadjust pre{\hypertarget{ref-dechow1997returns}{}}%
Dechow, Patricia M, and Richard G Sloan. 1997. {``Returns to Contrarian
Investment Strategies: Tests of Naive Expectations Hypotheses.''}
\emph{Journal of Financial Economics} 43 (1): 3--27.

\leavevmode\vadjust pre{\hypertarget{ref-dichev1998risk}{}}%
Dichev, Ilia D. 1998. {``Is the Risk of Bankruptcy a Systematic Risk?''}
\emph{The Journal of Finance} 53 (3): 1131--47.

\leavevmode\vadjust pre{\hypertarget{ref-diether2002differences}{}}%
Diether, Karl B, Christopher J Malloy, and Anna Scherbina. 2002.
{``Differences of Opinion and the Cross Section of Stock Returns.''}
\emph{The Journal of Finance} 57 (5): 2113--41.

\leavevmode\vadjust pre{\hypertarget{ref-dittmar2002nonlinear}{}}%
Dittmar, Robert F. 2002. {``Nonlinear Pricing Kernels, Kurtosis
Preference, and Evidence from the Cross Section of Equity Returns.''}
\emph{The Journal of Finance} 57 (1): 369--403.

\leavevmode\vadjust pre{\hypertarget{ref-doyle2003predictive}{}}%
Doyle, Jeffrey T, Russell J Lundholm, and Mark T Soliman. 2003. {``The
Predictive Value of Expenses Excluded from Pro Forma Earnings.''}
\emph{Review of Accounting Studies} 8: 145--74.

\leavevmode\vadjust pre{\hypertarget{ref-drake2011should}{}}%
Drake, Michael S, Lynn Rees, and Edward P Swanson. 2011. {``Should
Investors Follow the Prophets or the Bears? Evidence on the Use of
Public Information by Analysts and Short Sellers.''} \emph{The
Accounting Review} 86 (1): 101--30.

\leavevmode\vadjust pre{\hypertarget{ref-easley2010factoring}{}}%
Easley, David, Soeren Hvidkjaer, and Maureen O'hara. 2010. {``Factoring
Information into Returns.''} \emph{Journal of Financial and Quantitative
Analysis} 45 (2): 293--309.

\leavevmode\vadjust pre{\hypertarget{ref-eberhart2004examination}{}}%
Eberhart, Allan C, William F Maxwell, and Akhtar R Siddique. 2004. {``An
Examination of Long-Term Abnormal Stock Returns and Operating
Performance Following r\&d Increases.''} \emph{The Journal of Finance}
59 (2): 623--50.

\leavevmode\vadjust pre{\hypertarget{ref-eisfeldt2013organization}{}}%
Eisfeldt, Andrea L, and Dimitris Papanikolaou. 2013. {``Organization
Capital and the Cross-Section of Expected Returns.''} \emph{The Journal
of Finance} 68 (4): 1365--1406.

\leavevmode\vadjust pre{\hypertarget{ref-elgers2001delayed}{}}%
Elgers, Pieter T, May H Lo, and Ray J Pfeiffer Jr. 2001. {``Delayed
Security Price Adjustments to Financial Analysts' Forecasts of Annual
Earnings.''} \emph{The Accounting Review} 76 (4): 613--32.

\leavevmode\vadjust pre{\hypertarget{ref-fabozzi2008sin}{}}%
Fabozzi, Frank J, KC Ma, and Becky J Oliphant. 2008. {``Sin Stock
Returns.''} \emph{The Journal of Portfolio Management} 35 (1): 82--94.

\leavevmode\vadjust pre{\hypertarget{ref-fama1992cross}{}}%
Fama, Eugene F, and Kenneth R French. 1992. {``The Cross-Section of
Expected Stock Returns.''} \emph{The Journal of Finance} 47 (2):
427--65.

\leavevmode\vadjust pre{\hypertarget{ref-fama1993common}{}}%
---------. 1993. {``Common Risk Factors in the Returns on Stocks and
Bonds.''} \emph{Journal of Financial Economics} 33 (1): 3--56.

\leavevmode\vadjust pre{\hypertarget{ref-fama2006profitability}{}}%
---------. 2006. {``Profitability, Investment and Average Returns.''}
\emph{Journal of Financial Economics} 82 (3): 491--518.

\leavevmode\vadjust pre{\hypertarget{ref-fama1973risk}{}}%
Fama, Eugene F, and James D MacBeth. 1973. {``Risk, Return, and
Equilibrium: Empirical Tests.''} \emph{Journal of Political Economy} 81
(3): 607--36.

\leavevmode\vadjust pre{\hypertarget{ref-ferson1991variation}{}}%
Ferson, Wayne E, and Campbell R Harvey. 1991. {``The Variation of
Economic Risk Premiums.''} \emph{Journal of Political Economy} 99 (2):
385--415.

\leavevmode\vadjust pre{\hypertarget{ref-figlewski1981informational}{}}%
Figlewski, Stephen. 1981. {``The Informational Effects of Restrictions
on Short Sales: Some Empirical Evidence.''} \emph{Journal of Financial
and Quantitative Analysis} 16 (4): 463--76.

\leavevmode\vadjust pre{\hypertarget{ref-fishman1995mandatory}{}}%
Fishman, Micheal J, and Kathleen M Hagerty. 1995. {``The Mandatory
Disclosure of Trades and Market Liquidity.''} \emph{The Review of
Financial Studies} 8 (3): 637--76.

\leavevmode\vadjust pre{\hypertarget{ref-fogler1981three}{}}%
Fogler, H Russell, Rose John, and James Tipton. 1981. {``Three Factors,
Interest Rate Differentials and Stock Groups.''} \emph{The Journal of
Finance} 36 (2): 323--35.

\leavevmode\vadjust pre{\hypertarget{ref-foster1984earnings}{}}%
Foster, George, Chris Olsen, and Terry Shevlin. 1984. {``Earnings
Releases, Anomalies, and the Behavior of Security Returns.''}
\emph{Accounting Review}, 574--603.

\leavevmode\vadjust pre{\hypertarget{ref-frankel1998accounting}{}}%
Frankel, Richard, and Charles MC Lee. 1998. {``Accounting Valuation,
Market Expectation, and Cross-Sectional Stock Returns.''} \emph{Journal
of Accounting and Economics} 25 (3): 283--319.

\leavevmode\vadjust pre{\hypertarget{ref-franzoni2006pension}{}}%
Franzoni, Francesco, and Jose M Marin. 2006. {``Pension Plan Funding and
Stock Market Efficiency.''} \emph{The Journal of Finance} 61 (2):
921--56.

\leavevmode\vadjust pre{\hypertarget{ref-frazzini2014betting}{}}%
Frazzini, Andrea, and Lasse Heje Pedersen. 2014. {``Betting Against
Beta.''} \emph{Journal of Financial Economics} 111 (1): 1--25.

\leavevmode\vadjust pre{\hypertarget{ref-garlappi2008default}{}}%
Garlappi, Lorenzo, Tao Shu, and Hong Yan. 2008. {``Default Risk,
Shareholder Advantage, and Stock Returns.''} \emph{The Review of
Financial Studies} 21 (6): 2743--78.

\leavevmode\vadjust pre{\hypertarget{ref-george200452}{}}%
George, Thomas J, and Chuan-Yang Hwang. 2004. {``The 52-Week High and
Momentum Investing.''} \emph{The Journal of Finance} 59 (5): 2145--76.

\leavevmode\vadjust pre{\hypertarget{ref-gompers2001institutional}{}}%
Gompers, Paul A, and Andrew Metrick. 2001. {``Institutional Investors
and Equity Prices.''} \emph{The Quarterly Journal of Economics} 116 (1):
229--59.

\leavevmode\vadjust pre{\hypertarget{ref-gompers2003corporate}{}}%
Gompers, Paul, Joy Ishii, and Andrew Metrick. 2003. {``Corporate
Governance and Equity Prices.''} \emph{The Quarterly Journal of
Economics} 118 (1): 107--56.

\leavevmode\vadjust pre{\hypertarget{ref-gow2009earnings}{}}%
Gow, Ian D, and Daniel Taylor. 2009. {``Earnings Volatility and the
Cross-Section of Returns.''} \emph{Deloitte Foundation}. Working Paper.

\leavevmode\vadjust pre{\hypertarget{ref-griffin2002book}{}}%
Griffin, John M, and Michael L Lemmon. 2002. {``Book-to-Market Equity,
Distress Risk, and Stock Returns.''} \emph{The Journal of Finance} 57
(5): 2317--36.

\leavevmode\vadjust pre{\hypertarget{ref-guo2008average}{}}%
Guo, Hui, and Robert Savickas. 2008. {``Average Idiosyncratic Volatility
in G7 Countries.''} \emph{The Review of Financial Studies} 21 (3):
1259--96.

\leavevmode\vadjust pre{\hypertarget{ref-gupta1975investors}{}}%
Gupta, Manak C, and Aharon R Ofer. 1975. {``INVESTORS'EXPECTATIONS OF
EARNINGS GROWTH, THEIR ACCURACY AND EFFECTS ON THE STRUCTURE OF REALIZED
RATES OF RETURN.''} \emph{The Journal of Finance} 30 (2): 509--23.

\leavevmode\vadjust pre{\hypertarget{ref-hafzalla2011percent}{}}%
Hafzalla, Nader, Russell Lundholm, and E Matthew Van Winkle. 2011.
{``Percent Accruals.''} \emph{The Accounting Review} 86 (1): 209--36.

\leavevmode\vadjust pre{\hypertarget{ref-hahn2005financial}{}}%
Hahn, Jaehoon, and Hangyong Lee. 2005. {``Financial Constraints, Debt
Capacity, and the Cross Section of Stock Returns.''} \emph{Debt
Capacity, and the Cross Section of Stock Returns (May 2005)}.

\leavevmode\vadjust pre{\hypertarget{ref-harvey2000conditional}{}}%
Harvey, Campbell R, and Akhtar Siddique. 2000. {``Conditional Skewness
in Asset Pricing Tests.''} \emph{The Journal of Finance} 55 (3):
1263--95.

\leavevmode\vadjust pre{\hypertarget{ref-hawkins1984earnings}{}}%
Hawkins, Eugene H, Stanley C Chamberlin, and Wayne E Daniel. 1984.
{``Earnings Expectations and Security Prices.''} \emph{Financial
Analysts Journal} 40 (5): 24--38.

\leavevmode\vadjust pre{\hypertarget{ref-heckerman1972portfolio}{}}%
Heckerman, Donald G. 1972. {``Portfolio Selection and the Structure of
Capital Asset Prices When Relative Prices of Consumption Goods May
Change.''} \emph{The Journal of Finance} 27 (1): 47--60.

\leavevmode\vadjust pre{\hypertarget{ref-hirshleifer2013innovative}{}}%
Hirshleifer, David, Po-Hsuan Hsu, and Dongmei Li. 2013. {``Innovative
Efficiency and Stock Returns.''} \emph{Journal of Financial Economics}
107 (3): 632--54.

\leavevmode\vadjust pre{\hypertarget{ref-hirshleifer2010financing}{}}%
Hirshleifer, David, and Danling Jiang. 2010. {``A Financing-Based
Misvaluation Factor and the Cross-Section of Expected Returns.''}
\emph{The Review of Financial Studies} 23 (9): 3401--36.

\leavevmode\vadjust pre{\hypertarget{ref-holthausen1992prediction}{}}%
Holthausen, Robert W, and David F Larcker. 1992. {``The Prediction of
Stock Returns Using Financial Statement Information.''} \emph{Journal of
Accounting and Economics} 15 (2-3): 373--411.

\leavevmode\vadjust pre{\hypertarget{ref-hou2011factors}{}}%
Hou, Kewei, G Andrew Karolyi, and Bong-Chan Kho. 2011. {``What Factors
Drive Global Stock Returns?''} \emph{The Review of Financial Studies} 24
(8): 2527--74.

\leavevmode\vadjust pre{\hypertarget{ref-hou2005market}{}}%
Hou, Kewei, and Tobias J Moskowitz. 2005. {``Market Frictions, Price
Delay, and the Cross-Section of Expected Returns.''} \emph{The Review of
Financial Studies} 18 (3): 981--1020.

\leavevmode\vadjust pre{\hypertarget{ref-hou2015digesting}{}}%
Hou, Kewei, Chen Xue, and Lu Zhang. 2015. {``Digesting Anomalies: An
Investment Approach.''} \emph{The Review of Financial Studies} 28 (3):
650--705.

\leavevmode\vadjust pre{\hypertarget{ref-jagannathan1996conditional}{}}%
Jagannathan, Ravi, and Zhenyu Wang. 1996. {``The Conditional CAPM and
the Cross-Section of Expected Returns.''} \emph{The Journal of Finance}
51 (1): 3--53.

\leavevmode\vadjust pre{\hypertarget{ref-jegadeesh1990evidence}{}}%
Jegadeesh, Narasimhan. 1990. {``Evidence of Predictable Behavior of
Security Returns.''} \emph{The Journal of Finance} 45 (3): 881--98.

\leavevmode\vadjust pre{\hypertarget{ref-jegadeesh2004analyzing}{}}%
Jegadeesh, Narasimhan, Joonghyuk Kim, Susan D Krische, and Charles MC
Lee. 2004. {``Analyzing the Analysts: When Do Recommendations Add
Value?''} \emph{The Journal of Finance} 59 (3): 1083--1124.

\leavevmode\vadjust pre{\hypertarget{ref-jegadeesh1993returns}{}}%
Jegadeesh, Narasimhan, and Sheridan Titman. 1993. {``Returns to Buying
Winners and Selling Losers: Implications for Stock Market Efficiency.''}
\emph{The Journal of Finance} 48 (1): 65--91.

\leavevmode\vadjust pre{\hypertarget{ref-jiang2005information}{}}%
Jiang, Guohua, Charles MC Lee, and Yi Zhang. 2005. {``Information
Uncertainty and Expected Returns.''} \emph{Review of Accounting Studies}
10: 185--221.

\leavevmode\vadjust pre{\hypertarget{ref-johnson2012option}{}}%
Johnson, Travis L, and Eric C So. 2012. {``The Option to Stock Volume
Ratio and Future Returns.''} \emph{Journal of Financial Economics} 106
(2): 262--86.

\leavevmode\vadjust pre{\hypertarget{ref-kaplan1995financing}{}}%
Kaplan, Steven N, and Luigi Zingales. 1995. {``Do Financing Constraints
Explain Why Investment Is Correlated with Cash Flow?''} \emph{SSRN}.
National Bureau of economic research Cambridge, Mass., USA.

\leavevmode\vadjust pre{\hypertarget{ref-kelly2021understanding}{}}%
Kelly, Bryan T, Tobias J Moskowitz, and Seth Pruitt. 2021.
{``Understanding Momentum and Reversal.''} \emph{Journal of Financial
Economics} 140 (3): 726--43.

\leavevmode\vadjust pre{\hypertarget{ref-kishore2008earnings}{}}%
Kishore, Runeet, Michael W Brandt, Pedro Santa-Clara, and Mohan
Venkatachalam. 2008. {``Earnings Announcements Are Full of Surprises.''}
\emph{Available at SSRN 909563}.

\leavevmode\vadjust pre{\hypertarget{ref-kraus1976skewness}{}}%
Kraus, Alan, and Robert H Litzenberger. 1976. {``Skewness Preference and
the Valuation of Risk Assets.''} \emph{The Journal of Finance} 31 (4):
1085--1100.

\leavevmode\vadjust pre{\hypertarget{ref-kumar2008estimation}{}}%
Kumar, Praveen, Sorin M Sorescu, Rodney D Boehme, and Bartley R
Danielsen. 2008. {``Estimation Risk, Information, and the Conditional
CAPM: Theory and Evidence.''} \emph{The Review of Financial Studies} 21
(3): 1037--75.

\leavevmode\vadjust pre{\hypertarget{ref-la1996expectations}{}}%
La Porta, Rafael. 1996. {``Expectations and the Cross-Section of Stock
Returns.''} \emph{The Journal of Finance} 51 (5): 1715--42.

\leavevmode\vadjust pre{\hypertarget{ref-lamont2001financial}{}}%
Lamont, Owen, Christopher Polk, and Jesús Saaá-Requejo. 2001.
{``Financial Constraints and Stock Returns.''} \emph{The Review of
Financial Studies} 14 (2): 529--54.

\leavevmode\vadjust pre{\hypertarget{ref-landsman2011investors}{}}%
Landsman, Wayne R, Bruce L Miller, Ken Peasnell, and Shu Yeh. 2011.
{``Do Investors Understand Really Dirty Surplus?''} \emph{The Accounting
Review} 86 (1): 237--58.

\leavevmode\vadjust pre{\hypertarget{ref-lebedeva2012trading}{}}%
Lebedeva, Olga, Ernst Maug, and Christoph Schneider. 2012. {``Trading
Strategies of Corporate Insiders.''} \emph{Preprint}.

\leavevmode\vadjust pre{\hypertarget{ref-lee2000price}{}}%
Lee, Charles MC, and Bhaskaran Swaminathan. 2000. {``Price Momentum and
Trading Volume.''} \emph{The Journal of Finance} 55 (5): 2017--69.

\leavevmode\vadjust pre{\hypertarget{ref-lev2005r}{}}%
Lev, Baruch, Bharat Sarath, and Theodore Sougiannis. 2005. {``R\&d
Reporting Biases and Their Consequences.''} \emph{Contemporary
Accounting Research} 22 (4): 977--1026.

\leavevmode\vadjust pre{\hypertarget{ref-lev1996capitalization}{}}%
Lev, Baruch, and Theodore Sougiannis. 1996. {``The Capitalization,
Amortization, and Value-Relevance of r\&d.''} \emph{Journal of
Accounting and Economics} 21 (1): 107--38.

\leavevmode\vadjust pre{\hypertarget{ref-li2011well}{}}%
Li, Kevin Ke. 2011. {``How Well Do Investors Understand Loss
Persistence?''} \emph{Review of Accounting Studies} 16: 630--67.

\leavevmode\vadjust pre{\hypertarget{ref-lintner1965security}{}}%
Lintner, John. 1965. {``Security Prices, Risk, and Maximal Gains from
Diversification.''} \emph{The Journal of Finance} 20 (4): 587--615.

\leavevmode\vadjust pre{\hypertarget{ref-litzenberger1979effect}{}}%
Litzenberger, Robert H, and Krishna Ramaswamy. 1979. {``The Effect of
Personal Taxes and Dividends on Capital Asset Prices: Theory and
Empirical Evidence.''} \emph{Journal of Financial Economics} 7 (2):
163--95.

\leavevmode\vadjust pre{\hypertarget{ref-liu2006liquidity}{}}%
Liu, Weimin. 2006. {``A Liquidity-Augmented Capital Asset Pricing
Model.''} \emph{Journal of Financial Economics} 82 (3): 631--71.

\leavevmode\vadjust pre{\hypertarget{ref-lo2006trading}{}}%
Lo, Andrew W, and Jiang Wang. 2006. {``Trading Volume: Implications of
an Intertemporal Capital Asset Pricing Model.''} \emph{The Journal of
Finance} 61 (6): 2805--40.

\leavevmode\vadjust pre{\hypertarget{ref-loughran1995new}{}}%
Loughran, Tim, and Jay R Ritter. 1995. {``The New Issues Puzzle.''}
\emph{The Journal of Finance} 50 (1): 23--51.

\leavevmode\vadjust pre{\hypertarget{ref-mayshar1981transaction}{}}%
Mayshar, Joram. 1981. {``Transaction Costs and the Pricing of Assets.''}
\emph{The Journal of Finance} 36 (3): 583--97.

\leavevmode\vadjust pre{\hypertarget{ref-menzly2010market}{}}%
Menzly, Lior, and Oguzhan Ozbas. 2010. {``Market Segmentation and
Cross-Predictability of Returns.''} \emph{The Journal of Finance} 65
(4): 1555--80.

\leavevmode\vadjust pre{\hypertarget{ref-michaely1995price}{}}%
Michaely, Roni, Richard H Thaler, and Kent L Womack. 1995. {``Price
Reactions to Dividend Initiations and Omissions: Overreaction or
Drift?''} \emph{The Journal of Finance} 50 (2): 573--608.

\leavevmode\vadjust pre{\hypertarget{ref-mohanram2005separating}{}}%
Mohanram, Partha S. 2005. {``Separating Winners from Losers Among
Lowbook-to-Market Stocks Using Financial Statement Analysis.''}
\emph{Review of Accounting Studies} 10: 133--70.

\leavevmode\vadjust pre{\hypertarget{ref-moskowitz1999industries}{}}%
Moskowitz, Tobias J, and Mark Grinblatt. 1999. {``Do Industries Explain
Momentum?''} \emph{The Journal of Finance} 54 (4): 1249--90.

\leavevmode\vadjust pre{\hypertarget{ref-mossin1966equilibrium}{}}%
Mossin, Jan. 1966. {``Equilibrium in a Capital Asset Market.''}
\emph{Econometrica: Journal of the Econometric Society}, 768--83.

\leavevmode\vadjust pre{\hypertarget{ref-nagel2005short}{}}%
Nagel, Stefan. 2005. {``Short Sales, Institutional Investors and the
Cross-Section of Stock Returns.''} \emph{Journal of Financial Economics}
78 (2): 277--309.

\leavevmode\vadjust pre{\hypertarget{ref-narayanamoorthy2006conservatism}{}}%
Narayanamoorthy, Ganapathi. 2006. {``Conservatism and Cross-Sectional
Variation in the Post--Earnings Announcement Drift.''} \emph{Journal of
Accounting Research} 44 (4): 763--89.

\leavevmode\vadjust pre{\hypertarget{ref-novy2013other}{}}%
Novy-Marx, Robert. 2013. {``The Other Side of Value: The Gross
Profitability Premium.''} \emph{Journal of Financial Economics} 108 (1):
1--28.

\leavevmode\vadjust pre{\hypertarget{ref-ohlson1980financial}{}}%
Ohlson, James A. 1980. {``Financial Ratios and the Probabilistic
Prediction of Bankruptcy.''} \emph{Journal of Accounting Research},
109--31.

\leavevmode\vadjust pre{\hypertarget{ref-ou1989financial}{}}%
Ou, Jane A, and Stephen H Penman. 1989. {``Financial Statement Analysis
and the Prediction of Stock Returns.''} \emph{Journal of Accounting and
Economics} 11 (4): 295--329.

\leavevmode\vadjust pre{\hypertarget{ref-pastor2003liquidity}{}}%
P'astor, L'ubos, and Robert F Stambaugh. 2003. {``Liquidity Risk and
Expected Stock Returns.''} \emph{Journal of Political Economy} 111 (3):
642--85.

\leavevmode\vadjust pre{\hypertarget{ref-pagano1996transparency}{}}%
Pagano, Marco, and Ailsa R"oell. 1996. {``Transparency and Liquidity: A
Comparison of Auction and Dealer Markets with Informed Trading.''}
\emph{The Journal of Finance} 51 (2): 579--611.

\leavevmode\vadjust pre{\hypertarget{ref-palazzo2012cash}{}}%
Palazzo, Berardino. 2012. {``Cash Holdings, Risk, and Expected
Returns.''} \emph{Journal of Financial Economics} 104 (1): 162--85.

\leavevmode\vadjust pre{\hypertarget{ref-piotroski2000value}{}}%
Piotroski, Joseph D. 2000. {``Value Investing: The Use of Historical
Financial Statement Information to Separate Winners from Losers.''}
\emph{Journal of Accounting Research}, 1--41.

\leavevmode\vadjust pre{\hypertarget{ref-pontiff2008share}{}}%
Pontiff, Jeffrey, and Artemiza Woodgate. 2008. {``Share Issuance and
Cross-Sectional Returns.''} \emph{The Journal of Finance} 63 (2):
921--45.

\leavevmode\vadjust pre{\hypertarget{ref-prakash2013deferred}{}}%
Prakash, Rachna, and Nishi Sinha. 2013. {``Deferred Revenues and the
Matching of Revenues and Expenses.''} \emph{Contemporary Accounting
Research} 30 (2): 517--48.

\leavevmode\vadjust pre{\hypertarget{ref-rozeff1988market}{}}%
Rozeff, Michael S, and Mir A Zaman. 1988. {``Market Efficiency and
Insider Trading: New Evidence.''} \emph{Journal of Business}, 25--44.

\leavevmode\vadjust pre{\hypertarget{ref-sharpe1964capital}{}}%
Sharpe, William F. 1964. {``Capital Asset Prices: A Theory of Market
Equilibrium Under Conditions of Risk.''} \emph{The Journal of Finance}
19 (3): 425--42.

\leavevmode\vadjust pre{\hypertarget{ref-sloan1996stock}{}}%
Sloan, Richard G. 1996. {``Do Stock Prices Fully Reflect Information in
Accruals and Cash Flows about Future Earnings?''} \emph{Accounting
Review}, 289--315.

\leavevmode\vadjust pre{\hypertarget{ref-soliman2008use}{}}%
Soliman, Mark T. 2008. {``The Use of DuPont Analysis by Market
Participants.''} \emph{The Accounting Review} 83 (3): 823--53.

\leavevmode\vadjust pre{\hypertarget{ref-spiess1995underperformance}{}}%
Spiess, D Katherine, and John Affleck-Graves. 1995. {``Underperformance
in Long-Run Stock Returns Following Seasoned Equity Offerings.''}
\emph{Journal of Financial Economics} 38 (3): 243--67.

\leavevmode\vadjust pre{\hypertarget{ref-spiess1999long}{}}%
---------. 1999. {``The Long-Run Performance of Stock Returns Following
Debt Offerings.''} \emph{Journal of Financial Economics} 54 (1): 45--73.

\leavevmode\vadjust pre{\hypertarget{ref-thomas2011tax}{}}%
Thomas, Jacob, and Frank X Zhang. 2011. {``Tax Expense Momentum.''}
\emph{Journal of Accounting Research} 49 (3): 791--821.

\leavevmode\vadjust pre{\hypertarget{ref-titman2004capital}{}}%
Titman, Sheridan, KC John Wei, and Feixue Xie. 2004. {``Capital
Investments and Stock Returns.''} \emph{Journal of Financial and
Quantitative Analysis} 39 (4): 677--700.

\leavevmode\vadjust pre{\hypertarget{ref-tuzel2010corporate}{}}%
Tuzel, Selale. 2010. {``Corporate Real Estate Holdings and the
Cross-Section of Stock Returns.''} \emph{The Review of Financial
Studies} 23 (6): 2268--2302.

\leavevmode\vadjust pre{\hypertarget{ref-valta2016strategic}{}}%
Valta, Philip. 2016. {``Strategic Default, Debt Structure, and Stock
Returns.''} \emph{Journal of Financial and Quantitative Analysis} 51
(1): 197--229.

\leavevmode\vadjust pre{\hypertarget{ref-vanden2004options}{}}%
Vanden, Joel M. 2004. {``Options Trading and the CAPM.''} \emph{The
Review of Financial Studies} 17 (1): 207--38.

\leavevmode\vadjust pre{\hypertarget{ref-vanden2006option}{}}%
---------. 2006. {``Option Coskewness and Capital Asset Pricing.''}
\emph{The Review of Financial Studies} 19 (4): 1279--1320.

\leavevmode\vadjust pre{\hypertarget{ref-watkins2003riding}{}}%
Watkins, Boyce. 2003. {``Riding the Wave of Sentiment: An Analysis of
Return Consistency as a Predictor of Future Returns.''} \emph{The
Journal of Behavioral Finance} 4 (4): 191--200.

\leavevmode\vadjust pre{\hypertarget{ref-whited2006financial}{}}%
Whited, Toni M, and Guojun Wu. 2006. {``Financial Constraints Risk.''}
\emph{The Review of Financial Studies} 19 (2): 531--59.

\leavevmode\vadjust pre{\hypertarget{ref-womack1996brokerage}{}}%
Womack, Kent L. 1996. {``Do Brokerage Analysts' Recommendations Have
Investment Value?''} \emph{The Journal of Finance} 51 (1): 137--67.

\leavevmode\vadjust pre{\hypertarget{ref-xing2010does}{}}%
Xing, Yuhang, Xiaoyan Zhang, and Rui Zhao. 2010. {``What Does the
Individual Option Volatility Smirk Tell Us about Future Equity
Returns?''} \emph{Journal of Financial and Quantitative Analysis} 45
(3): 641--62.

\leavevmode\vadjust pre{\hypertarget{ref-yan2011jump}{}}%
Yan, Shu. 2011. {``Jump Risk, Stock Returns, and Slope of Implied
Volatility Smile.''} \emph{Journal of Financial Economics} 99 (1):
216--33.

\end{CSLReferences}



\end{document}
