% Options for packages loaded elsewhere
\PassOptionsToPackage{unicode}{hyperref}
\PassOptionsToPackage{hyphens}{url}
\PassOptionsToPackage{dvipsnames,svgnames,x11names}{xcolor}
%
\documentclass[
  letterpaper,
  DIV=11,
  numbers=noendperiod]{scrreprt}

\usepackage{amsmath,amssymb}
\usepackage{iftex}
\ifPDFTeX
  \usepackage[T1]{fontenc}
  \usepackage[utf8]{inputenc}
  \usepackage{textcomp} % provide euro and other symbols
\else % if luatex or xetex
  \usepackage{unicode-math}
  \defaultfontfeatures{Scale=MatchLowercase}
  \defaultfontfeatures[\rmfamily]{Ligatures=TeX,Scale=1}
\fi
\usepackage{lmodern}
\ifPDFTeX\else  
    % xetex/luatex font selection
\fi
% Use upquote if available, for straight quotes in verbatim environments
\IfFileExists{upquote.sty}{\usepackage{upquote}}{}
\IfFileExists{microtype.sty}{% use microtype if available
  \usepackage[]{microtype}
  \UseMicrotypeSet[protrusion]{basicmath} % disable protrusion for tt fonts
}{}
\makeatletter
\@ifundefined{KOMAClassName}{% if non-KOMA class
  \IfFileExists{parskip.sty}{%
    \usepackage{parskip}
  }{% else
    \setlength{\parindent}{0pt}
    \setlength{\parskip}{6pt plus 2pt minus 1pt}}
}{% if KOMA class
  \KOMAoptions{parskip=half}}
\makeatother
\usepackage{xcolor}
\setlength{\emergencystretch}{3em} % prevent overfull lines
\setcounter{secnumdepth}{5}
% Make \paragraph and \subparagraph free-standing
\ifx\paragraph\undefined\else
  \let\oldparagraph\paragraph
  \renewcommand{\paragraph}[1]{\oldparagraph{#1}\mbox{}}
\fi
\ifx\subparagraph\undefined\else
  \let\oldsubparagraph\subparagraph
  \renewcommand{\subparagraph}[1]{\oldsubparagraph{#1}\mbox{}}
\fi


\providecommand{\tightlist}{%
  \setlength{\itemsep}{0pt}\setlength{\parskip}{0pt}}\usepackage{longtable,booktabs,array}
\usepackage{calc} % for calculating minipage widths
% Correct order of tables after \paragraph or \subparagraph
\usepackage{etoolbox}
\makeatletter
\patchcmd\longtable{\par}{\if@noskipsec\mbox{}\fi\par}{}{}
\makeatother
% Allow footnotes in longtable head/foot
\IfFileExists{footnotehyper.sty}{\usepackage{footnotehyper}}{\usepackage{footnote}}
\makesavenoteenv{longtable}
\usepackage{graphicx}
\makeatletter
\def\maxwidth{\ifdim\Gin@nat@width>\linewidth\linewidth\else\Gin@nat@width\fi}
\def\maxheight{\ifdim\Gin@nat@height>\textheight\textheight\else\Gin@nat@height\fi}
\makeatother
% Scale images if necessary, so that they will not overflow the page
% margins by default, and it is still possible to overwrite the defaults
% using explicit options in \includegraphics[width, height, ...]{}
\setkeys{Gin}{width=\maxwidth,height=\maxheight,keepaspectratio}
% Set default figure placement to htbp
\makeatletter
\def\fps@figure{htbp}
\makeatother
\newlength{\cslhangindent}
\setlength{\cslhangindent}{1.5em}
\newlength{\csllabelwidth}
\setlength{\csllabelwidth}{3em}
\newlength{\cslentryspacingunit} % times entry-spacing
\setlength{\cslentryspacingunit}{\parskip}
\newenvironment{CSLReferences}[2] % #1 hanging-ident, #2 entry spacing
 {% don't indent paragraphs
  \setlength{\parindent}{0pt}
  % turn on hanging indent if param 1 is 1
  \ifodd #1
  \let\oldpar\par
  \def\par{\hangindent=\cslhangindent\oldpar}
  \fi
  % set entry spacing
  \setlength{\parskip}{#2\cslentryspacingunit}
 }%
 {}
\usepackage{calc}
\newcommand{\CSLBlock}[1]{#1\hfill\break}
\newcommand{\CSLLeftMargin}[1]{\parbox[t]{\csllabelwidth}{#1}}
\newcommand{\CSLRightInline}[1]{\parbox[t]{\linewidth - \csllabelwidth}{#1}\break}
\newcommand{\CSLIndent}[1]{\hspace{\cslhangindent}#1}

\usepackage{amsmath}
\KOMAoption{captions}{tableheading}
\makeatletter
\makeatother
\makeatletter
\@ifpackageloaded{bookmark}{}{\usepackage{bookmark}}
\makeatother
\makeatletter
\@ifpackageloaded{caption}{}{\usepackage{caption}}
\AtBeginDocument{%
\ifdefined\contentsname
  \renewcommand*\contentsname{Table of contents}
\else
  \newcommand\contentsname{Table of contents}
\fi
\ifdefined\listfigurename
  \renewcommand*\listfigurename{List of Figures}
\else
  \newcommand\listfigurename{List of Figures}
\fi
\ifdefined\listtablename
  \renewcommand*\listtablename{List of Tables}
\else
  \newcommand\listtablename{List of Tables}
\fi
\ifdefined\figurename
  \renewcommand*\figurename{Figure}
\else
  \newcommand\figurename{Figure}
\fi
\ifdefined\tablename
  \renewcommand*\tablename{Table}
\else
  \newcommand\tablename{Table}
\fi
}
\@ifpackageloaded{float}{}{\usepackage{float}}
\floatstyle{ruled}
\@ifundefined{c@chapter}{\newfloat{codelisting}{h}{lop}}{\newfloat{codelisting}{h}{lop}[chapter]}
\floatname{codelisting}{Listing}
\newcommand*\listoflistings{\listof{codelisting}{List of Listings}}
\makeatother
\makeatletter
\@ifpackageloaded{caption}{}{\usepackage{caption}}
\@ifpackageloaded{subcaption}{}{\usepackage{subcaption}}
\makeatother
\makeatletter
\@ifpackageloaded{tcolorbox}{}{\usepackage[skins,breakable]{tcolorbox}}
\makeatother
\makeatletter
\@ifundefined{shadecolor}{\definecolor{shadecolor}{rgb}{.97, .97, .97}}
\makeatother
\makeatletter
\makeatother
\makeatletter
\makeatother
\ifLuaTeX
  \usepackage{selnolig}  % disable illegal ligatures
\fi
\IfFileExists{bookmark.sty}{\usepackage{bookmark}}{\usepackage{hyperref}}
\IfFileExists{xurl.sty}{\usepackage{xurl}}{} % add URL line breaks if available
\urlstyle{same} % disable monospaced font for URLs
\hypersetup{
  pdftitle={Financial Anomalies},
  pdfauthor={Krishna Neupane},
  colorlinks=true,
  linkcolor={blue},
  filecolor={Maroon},
  citecolor={Blue},
  urlcolor={Blue},
  pdfcreator={LaTeX via pandoc}}

\title{Financial Anomalies}
\author{Krishna Neupane}
\date{2024-08-24}

\begin{document}
\maketitle
\ifdefined\Shaded\renewenvironment{Shaded}{\begin{tcolorbox}[enhanced, borderline west={3pt}{0pt}{shadecolor}, frame hidden, boxrule=0pt, interior hidden, breakable, sharp corners]}{\end{tcolorbox}}\fi

\renewcommand*\contentsname{Table of contents}
{
\hypersetup{linkcolor=}
\setcounter{tocdepth}{2}
\tableofcontents
}
\bookmarksetup{startatroot}

\hypertarget{preface}{%
\chapter*{Preface}\label{preface}}
\addcontentsline{toc}{chapter}{Preface}

\markboth{Preface}{Preface}

The article is desiged to study financial anomalies

\bookmarksetup{startatroot}

\hypertarget{introduction}{%
\chapter{Introduction}\label{introduction}}

Fama and MacBeth (1973) show two-parameter regression model estimates
average risk-return relationships based on efficient market porfolio
(\(m\)), that is, the market prices fully reflect the available
information. The asset are constructed based on Equation
\ref{Equation:BlackScholes1972} for an asset (\(i\)) proposed by @Black
(1972).

\begin{align}
    &x_{im}  \equiv \frac{\text{total market value of all units of assets} \ i}{\text{total market value of all assets}} \\
    &\text{where asset} (i) \text{in the portfolio} (m) \nonumber 
\end{align}
\label{Equation:BlackScholes1972}

Excepted return of a security (\(i\)) is \(E(\tilde{R_0})\), the
expected return on a security that is riskless in the portfolio \(m\),
plus a risk premium that is \(\beta_i\) times the difference between
expected return of the portfolio (\(E(\tilde{R_m})\) ) and riskless
portfolio (\(E(\tilde{R_0})\)). is calculated by Equation
\ref{Equation:ExpectedReturnFama1953}, \(\beta_i\) is the risk of the
asset \(i\) of the portfolio \(m\), measured relative to
\(\sigma^{2}(\tilde{R}_m)\)

\begin{align}
    &E(\tilde{R_i}) = \left[E(\tilde{R_m}) -S_m \sum\tilde{R_m}  \right] + S_m \sigma (\tilde{R_m}) \beta_i, \nonumber \\
    &\text{where}, \nonumber \\ 
    &\beta_i \equiv \frac{cov(\tilde{R_i}, \tilde{R_m})}{\sigma^2(\tilde{R_m})} = \frac{\sigma_{j=1}^{N} x_{jm}\sigma_{ij}}{\sigma^2 (\tilde{R_m})}=\frac{cov(\tilde{R_i},\tilde{R_m})/\sigma(\tilde{R_m})}{\sigma (\tilde{R_m})} \nonumber \\
    &S_m=\frac{E(\tilde{R_m})-E(\tilde{R_0})}{\sigma (\tilde{R_m})}  \nonumber\\
    \text{hence} \nonumber \\
    &E(\tilde{R_i}) = E(\tilde{R_0}) + \left[ E(\tilde{R_m}) - E(\tilde{R_0}) \right] \beta_i
\end{align}
\label{Equation:ExpectedReturnFama1953}

For each period of \(t\), the cross sectional regression is given by

\begin{align}
    & R_{pt} = \tilde{\gamma}_{0t} + \tilde{\gamma}_{1t} \tilde{\beta}_{p,t-1}+\tilde{\gamma}_{2t} \tilde{\beta}_{p,t-1}^{2} + \tilde{\gamma}_{3t} \bar{s}_{p, t-1} \tilde{\epsilon}_{i}+\tilde{\eta}_{pt}, \\
    &p=1,2,...t \nonumber
\end{align}
\label{Equation:ExpectedReturnFama1953EquationTen}

Equation \ref{Equation:ExpectedReturnFama1953EquationTen} the indepenent
variable \(\tilde{\beta}_{p,t-1}\) is the average of the
\(\tilde{\beta}_i\) for securities in portfolio \(p\),
\(\tilde{\beta}_{p,t-1}^2\) is the average of the squared values of
these \(\tilde{\beta}_i\), \(\bar{s}_{p, t-1} \tilde{\epsilon}_{i}\) is
the average of \(s\tilde{\epsilon}_i\) for portfolio \(p_i\)

Gupta and Ofer (1975) examines investors growth expectations reflected
in the stock prices. A change in the expectation is reflected in the
price movement. The study defines the earnings price ratio is a function
of risk characteristics of the security and the expected growth in the
earnings in Equation \ref{Equation:OferEquation1975EquationOne}. The
risk component are measured by: - the beta coefficient - the firm asset
size (natural logarithm of total asset) - dividend payout ratio -
leverage ratio of liabilties and preferred stocks to the common stock
outstanding - earnings variablity (standard deviation of earnings to
price ratio calcuated over period of seven years)

\begin{align}
    &EP = f(RS, EG) \\
    &\text{where:} \nonumber \\
    & EP = \text{earnings price ratio- }, \nonumber \\
    & RS = \text{risk characteristics of the security} \nonumber \\
    & EG =\text{the expected growth rate in the earnings} \nonumber \\
\end{align}
\label{Equation:OferEquation1975EquationOne}

\begin{align}
    & \Delta P_{i}^{t} = \frac{P_{it}-P_{it-1}}{P_{it-1}} \times 100 \\
    & \text{where:} \nonumber \\ 
    & \Delta P_{i}^{t} \text{is the percent change of the security } i \text{during the period } t-1 \text{to} t \nonumber \\
    & \text{The average yearly percentage of the prices change for portfolio } j (\Delta P_j) \text{ is given by} \nonumber \\
    & \Delta P_{j} =\frac{\sum_{t=1}^{14} \frac{\sum_{r_t=10j-9}^{10j} \Delta P_{rt}^{t}}{14} }{14}, r_{t}=1,\cdots,190 \nonumber\\
    & \text{where:} \nonumber \\
    & r_{t} = \text{the relative ranking of a security at time } t \text{ according to its prediction error at that year}. \\
    & \Delta P_{t}^{rt} = \text{percentage price change during the period } t-1 \text{ to } t \text{ for a security that has the rank of } r_t \text{ at time } t
\end{align}
\label{Equation:OferEquation1975}

Basu (1977): to determine empirically whether the investment performance
of the common stocks is related to the \(P/E\) ratios. \(P/E\) is the
ratio of market value of the common stock (market price times the number
of shares outstanding) as of December 31 to reporting annual earnings
(before extraordinary items) available for common stockholders.
According to the paper, due the exaggerated investor expectations,
\(P/E\) ratio may be the indicator of future investment peroformance.
The paper shows that \(P/E\) is not fully reflected in security prices
in as rapid a manner as postulated by the semi-strong form of the
efficient market hypothesis. Instead, it seems that disequilibrium
persisted in the capital market during the period studied. The results
suggests that market efficiency does not exist due to lags and frictions
in the price adjustmnet process.

Litzenberger and Ramaswamy (1979):

\begin{align}
    & \tilde{R} -r_{ft} = \gamma_{0} +  \gamma_{1} \beta_{it} +  \gamma_{2} (d_{it}-r_{ft}) + \tilde{\epsilon}_{it}, i=1,2,\cdots,N, t=1,2,\cdots,T
    &\text{litzenberger1979effect}
  
\end{align}
\label{Equation:Litzenberger1979Eqn23}

\bookmarksetup{startatroot}

\hypertarget{summary}{%
\chapter{Summary}\label{summary}}

In summary, this book has no content whatsoever.

\bookmarksetup{startatroot}

\hypertarget{references}{%
\chapter*{References}\label{references}}
\addcontentsline{toc}{chapter}{References}

\markboth{References}{References}

\hypertarget{refs}{}
\begin{CSLReferences}{1}{0}
\leavevmode\vadjust pre{\hypertarget{ref-basu1977investment}{}}%
Basu, Sanjoy. 1977. {``Investment Performance of Common Stocks in
Relation to Their Price-Earnings Ratios: A Test of the Efficient Market
Hypothesis.''} \emph{The Journal of Finance} 32 (3): 663--82.

\leavevmode\vadjust pre{\hypertarget{ref-black1972capital}{}}%
Black, Fischer. 1972. {``Capital Market Equilibrium with Restricted
Borrowing.''} \emph{The Journal of Business} 45 (3): 444--55.

\leavevmode\vadjust pre{\hypertarget{ref-fama1973risk}{}}%
Fama, Eugene F, and James D MacBeth. 1973. {``Risk, Return, and
Equilibrium: Empirical Tests.''} \emph{Journal of Political Economy} 81
(3): 607--36.

\leavevmode\vadjust pre{\hypertarget{ref-gupta1975investors}{}}%
Gupta, Manak C, and Aharon R Ofer. 1975. {``INVESTORS'EXPECTATIONS OF
EARNINGS GROWTH, THEIR ACCURACY AND EFFECTS ON THE STRUCTURE OF REALIZED
RATES OF RETURN.''} \emph{The Journal of Finance} 30 (2): 509--23.

\leavevmode\vadjust pre{\hypertarget{ref-litzenberger1979effect}{}}%
Litzenberger, Robert H, and Krishna Ramaswamy. 1979. {``The Effect of
Personal Taxes and Dividends on Capital Asset Prices: Theory and
Empirical Evidence.''} \emph{Journal of Financial Economics} 7 (2):
163--95.

\end{CSLReferences}



\end{document}
